\documentclass[12pt]{matemaj}
\usepackage[utf8]{inputenc}
\usepackage[italian]{babel}

\begin{document}
\titolo Sweet saturday

Cinque compagni di classe, Alice, Beatrice, Carlo, Davide ed Enrico vogliono trascorrere il sabato insieme andando a vedere un film al cinema ``Paradiso'' e andando a cena in pizzeria da ``Luigi''.

\quesito[10;30]
Un menu completo da ``Luigi'' ha un costo di 10 euro. Tuttavia, il sabato, Luigi propone la seguente offerta:\\[6pt]
{\bf Sconto comitiva per studenti: }{\it Il primo studente della comitiva paga a prezzo pieno, il secondo paga il 90\% di quanto ha pagato il primo, il terzo paga il 90\% di quanto ha pagato il secondo e così via...}\\[6pt]
I cinque compagni decidono di approfittare dell'offerta dividendo equamente il costo complessivo dei 5 menu. In questo modo quanto avrà effettivamente risparmiato ogni studente in percentuale rispetto al prezzo intero del menu?

I cinque compagni devono inoltre decidere se andare prima al cinema e poi a cena o viceversa. Per tale ragione si recano al cinema ``Paradiso'' per informarsi sull'orario di chiusura.

\quesito[0;100]
Nel cinema ``Paradiso'' sono presenti 3 sale nelle quali vengono proiettati 3 film diversi simultaneamente. All'entrata del cinema è presente un'insegna sulla quale non è indicato l'orario di chiusura ma sulla quale sono indicate le seguenti informazioni:\\[6pt]
{\bf Orario di apertura: }{\it 14:00}\\[6pt]
{\bf Sala 1: }{Le avventure di Archimede (durata: 1:15)}\\[6pt]
{\bf Sala 2: }{A cena con Euclide (durata: 1:20)}\\[6pt]
{\bf Sala 3: }{Pitagora all'improvviso (durata: 1:40)}\\[6pt]
Le durate indicate sono comprensive degli intervalli tra una proiezione e la successiva. All'apertura del cinema tutte le proiezioni iniziano simultaneamente e le ultime proiezioni dei film iniziano tra le 20:00 e le 21:00. Quanti minuti dopo le 21:00 il cinema avrà terminato tutte le proiezioni?

I cinque compagni decidono di andare prima al cinema ad assistere alla proiezione delle ore 17:45 de ``Le avventure di Archimede''.

\quesito[1;60]
Alle 17:45 il cinema non è molto affollato, così, i cinque compagni riescono a occupare 5 posti in fila. Sapendo che Alice vuole sedersi vicina a Carlo, in quanti differenti modi i compagni potranno disporsi nei 5 posti?

\quesitotc[200;300]
[.7]Usciti dalla sala, i cinque compagni si accorgono di una decorazione azzurra a forma di quadrifoglio su tutto il pavimento della grande sala all'entrata. Guardando con attenzione Enrico osserva che le 4 foglie del quadrifoglio si ottengono dall'intreccio di 4 semicerchi identici con il diametro coincidente con i 4 lati di un quadrato. Beatrice misura quindi il lato del grande quadrato contenente il quadrifoglio che risulta pari a 20 m. Alice intuisce che sfruttando solamente questo dato è possibile calcolare l'area del quadrifoglio. Quanto vale l'area del quadrifoglio azzurro espressa in metri quadrati?
[.3]\tikspace\begin{tikzpicture}[scale=.8]
\draw[clip] (0,0) rectangle (5,5);
\draw[thick] (2.5,0) circle (2.5cm);
\draw[thick] (0,2.5) circle (2.5cm);
\draw[thick] (2.5,5) circle (2.5cm);
\draw[thick] (5,2.5) circle (2.5cm);
\begin{scope}
    \path[clip] (2.5,0) circle (2.5cm);
    \path[clip] (0,2.5) circle (2.5cm);
    \fill[blue!50] (0,0) rectangle (5,5);
\end{scope}
\begin{scope}
    \path[clip] (0,2.5) circle (2.5cm);
    \path[clip] (2.5,5) circle (2.5cm);
    \fill[blue!50] (0,0) rectangle (5,5);
\end{scope}
\begin{scope}
    \path[clip] (2.5,5) circle (2.5cm);
    \path[clip] (5,2.5) circle (2.5cm);
    \fill[blue!50] (0,0) rectangle (5,5);
\end{scope}
\begin{scope}
    \path[clip] (5,2.5) circle (2.5cm);
    \path[clip] (2.5,0) circle (2.5cm);
    \fill[blue!50] (0,0) rectangle (5,5);
\end{scope}
\end{tikzpicture}

\newpage

\titolo Contact crash

Albert, nel tragitto verso la sua scuola, lascia distrattamente cadere il suo zaino a terra. Per timore che il conenuto dello zaino abbia subito danni, appena arriva a scuola, ripone tale contenuto sul banco per verificare che sia tutto integro. I primi oggetti che Albert estrae dallo zaino sono una squadra e il goniometro.
\begin{center}
\begin{tikzpicture}
   \draw[thick,fill=blue!30] (0,0) -- (4,0) -- (0,4) -- cycle;
   \draw[thick,fill=white] (0.5,0.5) -- (2.8,0.5) -- (0.5,2.8) -- cycle;
   \foreach \A in {1,...,15} {\draw[thick] (\A/4,0) -- (\A/4,.2);}
   \foreach \A in {1,...,15} {\draw[thick] (0,\A/4) -- (.2,\A/4);}
\end{tikzpicture}
\begin{tikzpicture}
   \draw[thick,fill=orange!30] (0,0) circle (2cm);
   \draw[thick,fill=white] (0,0) circle (1.5cm);
   \foreach \A in {0,10,...,350} {\draw[thick] ({1.7*cos(\A)},{1.7*sin(\A)}) -- ({2*cos(\A)},{2*sin(\A)});}
\end{tikzpicture}
\end{center}
A un primo sguardo gli oggetti sembrano integri ma, osservando da vicino, Albert si rende conto che entrambi gli oggetti sono lesionati.

\quesitotc[20;30]
[.65]Su ciascuno dei cateti della squadra è rappresentata una scala con tacche lunghe a intervalli di 1 cm e tacche corte a intervalli di 1 mm. Lo zero di entrambe le scale si trova nel vertice retto. Purtroppo la squadra presenta due lesioni perpendicolari (rappresentate in figura con delle linee blu). Qual è la misura della lesione più corta in millimetri?
[.35]\tikspace\begin{tikzpicture}[scale=3]
   \path[clip] (0,0) rectangle (1.35,.85);
   \draw[thick,fill=blue!30] (0,0) -- (4,0) -- (0,4) -- cycle;
   \draw[thick,fill=white] (0.5,0.5) -- (2.8,0.5) -- (0.5,2.8) -- cycle;
   \foreach \A in {5,...,60} {\draw (\A/40,0) -- (\A/40,.1);}
   \draw[thick] (1/4,0) -- (1/4,.2);
   \foreach \A in {2,...,6} {\draw[thick] (\A/4,0) -- (\A/4,.2) node[pos=1.4] {\A};}
   \foreach \A in {5,...,40} {\draw (0,\A/40) -- (.1,\A/40);}
   \draw[thick] (0,1/4) -- (.2,1/4);
   \foreach \A in {2,...,4} {\draw[thick] (0,\A/4) -- (.2,\A/4) node[pos=1.4] {\A};}
   \draw[blue, very thick] (1,0) -- (0,.75);
   \path[clip] (0,0) -- (1,0) -- (0,.75) -- cycle;
   \draw[blue,very thick] (0,0) -- (.75,1);   
\end{tikzpicture}

\vspace{-.8cm}
\quesitotc[350;400]
[.65]I due frammenti triangolari successivamente si distaccano dalla squadre. Quanto vale l'area del frammento più grande espressa in millimetri quadrati?
[.35]\tikspace\begin{tikzpicture}[scale=3]
   \path[clip] (0,0) -- (.36,.48) -- (0,.75) -- cycle;
   \draw[thick,fill=blue!30] (0,0) -- (4,0) -- (0,4) -- cycle;
   \draw[thick,fill=white] (0.5,0.5) -- (2.8,0.5) -- (0.5,2.8) -- cycle;
   \foreach \A in {5,...,60} {\draw (\A/40,0) -- (\A/40,.1);}
   \draw[thick] (1/4,0) -- (1/4,.2);
   \foreach \A in {2,...,6} {\draw[thick] (\A/4,0) -- (\A/4,.2) node[pos=1.4] {\A};}
   \foreach \A in {5,...,40} {\draw (0,\A/40) -- (.1,\A/40);}
   \draw[thick] (0,1/4) -- (.2,1/4);
   \foreach \A in {2,...,4} {\draw[thick] (0,\A/4) -- (.2,\A/4) node[pos=1.4] {\A};}
   \draw[blue,thick] (1,0) -- (0,.75);
   \path[clip] (0,0) -- (1,0) -- (0,.75) -- cycle;
   \draw[blue,thick] (0,0) -- (.75,1);   
\end{tikzpicture}
\begin{tikzpicture}[scale=3]
   \path[clip] (0,0) -- (.36,.48) -- (1,0) -- cycle;
   \draw[thick,fill=blue!30] (0,0) -- (4,0) -- (0,4) -- cycle;
   \draw[thick,fill=white] (0.5,0.5) -- (2.8,0.5) -- (0.5,2.8) -- cycle;
   \foreach \A in {5,...,60} {\draw (\A/40,0) -- (\A/40,.1);}
   \draw[thick] (1/4,0) -- (1/4,.2);
   \foreach \A in {2,...,6} {\draw[thick] (\A/4,0) -- (\A/4,.2) node[pos=1.4] {\A};}
   \foreach \A in {5,...,40} {\draw (0,\A/40) -- (.1,\A/40);}
   \draw[thick] (0,1/4) -- (.2,1/4);
   \foreach \A in {2,...,4} {\draw[thick] (0,\A/4) -- (.2,\A/4) node[pos=1.4] {\A};}
   \draw[blue,thick] (1,0) -- (0,.75);
   \path[clip] (0,0) -- (1,0) -- (0,.75) -- cycle;
   \draw[blue,thick] (0,0) -- (.75,1);   
\end{tikzpicture}

\vspace{-.8cm}
\quesitotc[300;400]
[.5]Sul goniometro è rappresentata una scala per misurare gli angoli sulla quale è riportata una tacca ogni 10$^\circ$. Purtroppo anche dal goniometro si è distaccato un frammento. Sapendo che il diametro del goniometro è 12 cm, quanto vale l'area del frammento distaccato espressa in millimetri quadrati?
[.5]\tikspace\begin{tikzpicture}[scale=3]
   \path[clip] (-1.3,{2*sin(60)}) rectangle (1.3,2.04);
   \draw[very thick,red,fill=orange!30] (0,0) circle (2cm);
   \draw[thick,fill=white] (0,0) circle (1.5cm);
   \foreach \A in {0,10,...,350} {\draw[thick] ({1.8*cos(\A)},{1.8*sin(\A)}) -- ({2*cos(\A)},{2*sin(\A)}) node[pos=-.5] {\A$^\circ$};}
   \draw[very thick,red] ({2*cos(120)},{2*sin(120)+0.003}) -- ({2*cos(60)},{2*sin(60)+0.003});   
\end{tikzpicture} \\ \vspace{.3cm}\tikspace\begin{tikzpicture}[scale=3]
   \path[clip] (-1.3,1) rectangle (1.3,{2*sin(60)});
   \draw[thick,fill=orange!30] (0,0) circle (2cm);
   \draw[thick,fill=white] (0,0) circle (1.5cm);
   \foreach \A in {0,10,...,350} {\draw[thick] ({1.8*cos(\A)},{1.8*sin(\A)}) -- ({2*cos(\A)},{2*sin(\A)}) node[pos=-.5] {\A$^\circ$};}
   \draw[very thick] ({2*cos(120)},{2*sin(120)-0.003}) -- ({2*cos(60)},{2*sin(60)-0.003});   
\end{tikzpicture}

\vspace{-.8cm}
\quesito[1;50]
Anche la calcolatrice è distrutta. Sono saltati quasi tutti i tasti. Gli unici tasti ancora funzionanti sono\; \btn{7} \btn{9} \btn{+} \btn{-} \btn{=}. Albert però si rende conto che, anche usando soltanto questi tasti, è possibile scrivere tutti i numeri interi. Per esempio, premendo la sequenza di tasti \btn{9} \btn{-} \btn{7} \btn{=} si ottiene il numero 2. Per ottenere il numero 2 è sufficiente quindi una sequenza di 4 tasti. Scrivere il numero 1 però è più complicato. Quanti tasti bisognerà premere almeno per ottenere il numero 1?

\newpage
\titolo Busy bugs

\begin{minipage}[c]{.5\textwidth}
Le formiche che vivono nel giardino di Richard hanno avvistata una montagna di zucchero nei pressi del loro formicaio. Decidono quindi di organizzare una spedizione per andare a prelevare lo zucchero. Ci tengono molto all'organizzazione così, prima di uscire dal formicaio, pianificano il percorso. Purtroppo la recente pioggia ha lasciato delle pozze di acqua, 5 delle quali di forma quadrata. Le formiche devono decidere quale sia la via migliore da seguire. Realizzano così una mappa e comprendono che il percorso più breve è costituito da un tratto rettilineo lungo 4 dm, un tratto a forma di arco di circonferenza di raggio 2 dm e un altro tratto rettilineo lungo 4~dm come indicato con la linea rossa tratteggiata sulla mappa. 
\end{minipage}\hspace{.05\textwidth}
\begin{minipage}[c]{.45\textwidth}
\begin{tikzpicture}[scale=2]
   \path[fill=brown!50] (0,0) circle (.3 cm) --(0,-.4) node {Zucchero};
   \path[fill=brown!50] (3,3) circle (.3 cm) -- (3,3.4) node {Formicaio};
   \begin{scope}
   \path[clip] (-0.1,-0.1) -- (3.1,-0.1) -- (3.1,3.1) -- (1,3.1) arc[start angle=90, delta angle=90, radius=1.1] -- cycle;
   \fill[brown!50] (-0.1,-0.1) rectangle (3.1,3.1);
   \path[clip] (0.1,0.1) -- (2.9,0.1) -- (2.9,2.9) -- (1,2.9) arc[start angle=90, delta angle=90, radius=0.9] -- cycle;
   \foreach \A in {0,...,2} {\foreach \B in {0,...,2} {
       \path[fill=blue!40] (\A+0.1,\B+0.1) rectangle (\A+0.9,\B+0.9);
   }}
   \path[fill=blue!40] (0.1,0.1) rectangle (1.9,1.9)  node[pos=0.5] {Acqua};
   \end{scope}
   %\draw[red,very thick] (0,0) -- (0,2) -- (1,2) -- (1,3) -- (2,3) -- (2,1) -- (3,1) -- (3,3);
   \path[fill=white] (0,0) circle (.1 cm);
   \path[fill=black] (3,3) circle (.1 cm);
   \draw[red,very thick,<-,dashed] (0,0) -- (0,2) arc[start angle=180, delta angle=-90, radius=1] -- (3,3);
   \draw[|-|,very thick] (0,3.3) -- (1,3.3) node[above,pos=.5] {2 dm};
   \draw[|-|,very thick] (1,3.3) -- (2,3.3) node[above,pos=.5] {2 dm};
   \draw[|-|,very thick] (-0.5,0) -- (-0.5,2) node[above,pos=.5,sloped] {4 dm};
   \draw[|-|,very thick] (-0.5,2) -- (-0.5,3) node[above,pos=.5,sloped] {2 dm};
   \draw[|-|,very thick] (3.5,1) -- (3.5,2) node[below,pos=.5,sloped] {2 dm};
   \draw[|-|,very thick] (3.5,0) -- (3.5,1) node[below,pos=.5,sloped] {2 dm};
\end{tikzpicture}
\end{minipage}

\quesito[20;30]
Quanti secondi impiegherà una formica per raggiungere la montagna di zucchero seguendo il percorso pianificato, sapendo che in assenza delle pozze d'acqua, alla stessa velocità, avrebbe impiegato 20 secondi?

Arritave alla montagna di zucchero, le formiche si rendono conto che lo zucchero è tanto e richiederà la collaborazione di tutte per essere trasportato. Tra di loro ci sono\\
formiche dette {\bf ``forzute''} capaci di trasportare 5 grani di zucchero per volta;\\
formiche dette {\bf ``così così''} capaci di trasportare 3 grani di zucchero per volta;\\
formiche dette {\bf ``scansafatiche''} capaci di trasportare 1 grano di zucchero per volta.\\
Se a trasportare lo zucchero fossero le sole formiche {\it forzute} servirebbero esattamente 3 viaggi, se fossero le sole formiche {\it così così} servirebbero esattamente 4 viaggi, se fossero le sole formiche {\it scansafatiche} servirebbero esattamente 5 viaggi. Nel formicaio ci sono complessivamente 105 formiche.

\quesito[200;400]
Quanti sono i grani di zucchero da trasportare?

\quesito[200;400]
Quanti grani di zucchero riusciranno a trasportare, collaborando tutte, in un solo viaggio?

\begin{minipage}[c]{.75\textwidth}
Le formiche, temendo che al loro ritorno dalla montagna di zucchero verso il formicaio, potessero trovare qualche via interrotta, hanno studiato tutti i possibili percorsi alternativi che permettessero di ritornare al formicaio come, ad esempio, quello indicato in figura con la linea rossa. \`E importante che la linea del possibile percorso non si intrecci con se stessa in alcun punto altrimenti, in tale punto, le formiche finirebbero per scontrarsi.
\end{minipage}\hspace{.05\textwidth}
\begin{minipage}[c]{.20\textwidth}
\begin{tikzpicture}
   \path[fill=brown!50] (0,0) circle (.3 cm) --(0,-.4);
   \path[fill=brown!50] (3,3) circle (.3 cm) -- (3,3.4);
   \begin{scope}
   \path[clip] (-0.1,-0.1) -- (3.1,-0.1) -- (3.1,3.1) -- (1,3.1) arc[start angle=90, delta angle=90, radius=1.1] -- cycle;
   \fill[brown!50] (-0.1,-0.1) rectangle (3.1,3.1);
   \path[clip] (0.1,0.1) -- (2.9,0.1) -- (2.9,2.9) -- (1,2.9) arc[start angle=90, delta angle=90, radius=0.9] -- cycle;
   \foreach \A in {0,...,2} {\foreach \B in {0,...,2} {
       \path[fill=blue!40] (\A+0.1,\B+0.1) rectangle (\A+0.9,\B+0.9);
   }}
   \path[fill=blue!40] (0.1,0.1) rectangle (1.9,1.9);
   \end{scope}
   \path[fill=white] (0,0) circle (.1 cm);
   \path[fill=black] (3,3) circle (.1 cm);
   \draw[red,very thick,->,dashed] (0,0) -- (0,2) -- (1,2) -- (1,3) -- (2,3) -- (2,1) -- (3,1) -- (3,3);
\end{tikzpicture}
\end{minipage}

\quesito[1;50]
Quanti sono i possibili percorsi, che non si intrecciano, che permettono alle formiche di ritornare al formicaio?

\newpage

\titolo Roaring rails

\vspace{-.6cm}
\begin{minipage}[c]{.4\textwidth}
Ogni giorno Albert e Richard si incontrano alle 7 in punto alla stazione {\bf A} della metropolitana. Entrambi lavorano in un importante labaoratorio che si trova nei pressi della stazione {\bf F} e perciò ogni giorno trascorrono il tempo del viaggio insieme.
\end{minipage}\hspace{.02\textwidth}
\begin{minipage}[c]{.6\textwidth}
\begin{tikzpicture}
   \foreach \A in {-4,...,4} {\draw[lightgray,dashed] (\A,-4.5) -- (\A,4.5);}
   \foreach \A in {-3,...,3} {\draw[lightgray,dashed] (-5.5,\A) -- (5.5,\A);}
   \draw[|-|,very thick] (-4,4.5) -- (-3,4.5) node[above,pos=.5] {2 km};
   \draw[|-|,very thick] (-5.5,2) -- (-5.5,3) node[above,sloped,pos=.5] {2 km};
   \foreach \A in {-5,...,4} {\foreach \B in {-4,...,3} {\fill[lightgray!50] (\A+.1,\B+.1) rectangle (\A+.9,\B+.9);}}
   \draw[line width=.3cm, red!40, rounded corners,text=black] (0,0) -- (0,-3) -- (4,-3) node[pos=.5,below,fill] {Linea rossa} -- cycle;
   \draw[line width=.3cm, blue!40, rounded corners, text=black] (0,0) -- (-2,-1.5) -- (-4,0) -- (-2,1.5) node[pos=.5,above,sloped,fill] {Linea blu} -- cycle;
   \draw[line width=.3cm, green!40, rounded corners, text=black] (0,0) -- (4,3) -- (4,0) -- cycle node[pos=.4,above,fill] {Linea verde};
   \draw[line width=.3cm, orange!40, rounded corners, text=black] (-2,-1.5) -- (-4,-1.5) -- (-4,-3) -- (0,-3) node[pos=.5,below,fill] {Linea arancio};
   \draw[line width=.3cm, magenta!40, rounded corners, text=black] (-2,1.5) -- (-2,3) -- (4,3) node[pos=.5,above,fill] {Linea magenta};
   \draw[very thick, red] (0,0) -- (0,-3) -- (4,-3) -- cycle;
   \draw[very thick, blue] (0,0) -- (-2,-1.5) -- (-4,0) -- (-2,1.5) -- cycle;
   \draw[very thick, green] (0,0) -- (4,3) -- (4,0) -- cycle;
   \draw[very thick, orange] (-2,-1.5) -- (-4,-1.5) -- (-4,-3) -- (0,-3);
   \draw[very thick, magenta] (-2,1.5) -- (-2,3) -- (4,3);
   \tikzstyle{metro}=[fill=red, text=white, minimum size=.6cm, circle]
   \node[metro] at (0,0) {C};
   \node[metro] at (-4,0) {A};
   \node[metro] at (-2,-1.5) {B};
   \node[metro] at (-2,1.5) {D};
   \node[metro] at (4,0) {E};
   \node[metro] at (4,3) {F};
   \node[metro] at (0,-3) {G};
   \node[metro] at (4,-3) {H};
   \node[metro] at (-2,3) {I};
   \node[metro] at (-4,-1.5) {J};
   \node[metro] at (-4,-3) {K};
\end{tikzpicture}
\end{minipage}

\def\posto(#1,#2){\fill[rounded corners, blue] (#1+.1,#2+.1) rectangle (#1+.9,#2+.9)}
\vspace{-.3cm}
\quesitotc[10;100]
[.6]Stamattina nella carrozza della metropolitana nella quale sono saliti Albert e Richard, tutti i posti a sedere sono liberi. Ci sono un gruppo di 5 posti, un gruppo di 4 posti, un gruppo di 3 posti, un gruppo di 2 posti e un posto singolo disposti come in figura. Albert e Richard vogliono sedersi nello stesso gruppo di posti per poter conversare. In quanti diversi modi i due potranno sedersi nella carrozza?\vspace{-.3cm}
[.4]\tikspace\begin{tikzpicture}[scale=.5]
   \node[above] at (5.5,5) {Mappa della carrozza};
   \draw[rounded corners] (0,0) rectangle (11,5);
   \draw[very thick] (1.7,5) -- (1.7,4);
   \draw[very thick] (1.7,0) -- (1.7,1);
   \draw[very thick] (5.3,5) -- (5.3,4);
   \draw[very thick] (4.3,0) -- (4.3,1);
   \draw[very thick] (4.3,1) arc[start angle=90, delta angle=-90, radius=1];
   \draw[very thick] (5.3,0)  arc[start angle=180, delta angle=-90, radius=1] -- (6.3,0);
   \draw[very thick] (8,0)  arc[start angle=180, delta angle=-90, radius=3];
   \node[below] at (5.3,0) {porta};
   \posto(0,0); \posto(0,1); \posto(0,2); \posto(0,3); \posto(0,4);
   \posto(2,0); \posto(3,0);
   \posto(2,4); \posto(3,4); \posto (4,4);
   \posto(6.6,0);
   \posto(5.6,4); \posto(6.6,4); \posto(7.6,4); \posto(8.6,4);
\end{tikzpicture}\vspace{-.3cm}

\vspace{-.8cm}
\quesito[100;200]
In quanti diversi modi Albert e Richard potrebbero invece sedersi in gruppi di posti diversi?

Le linee blu, rossa e verde sono delle linee ``circolari'', vale a dire compiono ripetutamente dei percorsi chiusi. La mattina le tre carrozze delle tre linee compiono la prima corsa della giornata partendo contemporaneamente dalla stazione {\bf C} alle ore 7:00. La carrozza della linea rossa impiega 15 minuti per compiere un giro completo, mentre quelle della linea verde e della linea blu ne impiegano 24. Alle ore 21:00 la metropolitana chiude.

\quesito[1;20]
Quante volte le carrozze delle tre linee si troveranno simultaneamente alla stazione {\bf C} nell'arco della giornata?

\vspace{-.8cm}
\quesito[1;24]
Per arrivare alla stazione {\bf F}, partendo dalla stazione {\bf A}, Albert e Richard possono scegliere due possibili percorsi: possono viaggiare con la linea blu fino alla stazione {\bf D} e dalla stazione {\bf D} proseguire con la linea magenta fino alla stazione {\bf F} (percorso {\bf A$\rightarrow$D$\rightarrow$I$\rightarrow$F}) oppure possono viaggiare con la linea blu fino alla stazione {\bf C} e della stazione {\bf C} proseguire con la linea verde fino alla stazione {\bf F} (percorso {\bf A$\rightarrow$D$\rightarrow$C$\rightarrow$F}). I due, che sono abituati a fare calcoli per mestiere, osservando la mappa della metropolitana si rendono conto che tali percorsi hanno esattamente la stessa lunghezza. Trascurando il tempo di permanenza delle carrozze nelle stazioni, considerando le velocità alle quali viaggiano le carrozze di tutte le linee costanti, quanti minuti dovrebbe impiegare la carrozza della linea magenta per percorrere il tratto {\bf D$\rightarrow$I$\rightarrow$F} in modo da garantire ad Albert e Richard di poter arrivare al lavoro nello stesso tempo, indipendentemente dal percorso scelto?

\end{document}
